	\chapter*{مقدمه}\label{ch-0}
	\section{کپی رایت}\label{se-01}
	تاریخ تهیه این آموزش در سال 1396 انجام شده است و تمام حقوق آن متعلق به سایت ذیل است
\begin{flushleft}
\href{www.linuxfedora.ir}{www.linuxfedora.ir}
\end{flushleft}	

هرگونه کپی برداری بدون هماهنگی با نویسنده ممنوع و حلاف شرع است

	\section{بازخورد}\label{se-02}
	لطفا هر گونه نظر یا پیشنهادات در مورد این آموزش را از طریق ایمیل  یا  ای دی تلگرام زیر برای من ارسال نمایید:
		
		گروه پرسش و پاسخ همین کتاب در پیام رسان گپ
		\href{https://Gap.im/gnu\_linux}{https://Gap.im/gnu\_linux}
		
		گروه پرسش و پاسخ فدورا در تلگرام
		\href{https://t.me/joinchat/DSbv30PVmG2lrcJjsAhwCQ}{http://opizo.com/pPkVlQ}
		
	کانال اختصاصی این کتاب در پیام رسان گپ
		\href{https://Gap.im/itmans}{https://Gap.im/itmans}
		
	\begin{flushleft}
		alireza@jahanifard.ir
		
		\href{https://gap.im/fedora}{@Fedora}

	\end{flushleft}
	\textbf{توجه:}در صورت غلط املایی  یا همکاری در نسخه بعدی لطفا با بنده مکاتبه داشته باشید
	\section{تشکر و قدردانی}\label{se-03}
	این کتاب تماما توسط لاتک تهیه گردیده است و باید تشکر ویژه ای از استادان  {\nast آقای دکتر علی مس فروش }و  {\nast آقای مددپور }بکنم که راهنمایی های بسیار سازنده برای تدوین این کتاب توسط لاتک به بنده حقیر نمودند
	\section{تاریخ انتشار و نسخه نرم افزار}\label{se-04}
این کتاب به شکلی آماده شده است که محدود به نسخه ای خاص نیست ولی پیشنهاد میکنیم از نسخه ۲۷ به بعد استفاده کنید
	\section{مقدمه}\label{se-05}
\index{Fedora}
تنها یک سیستم عامل آزاد و متن باز با حدود ۲ میلیون کاربر \cite{sta-fedora}
درسرتاسر جهان نیست; فدورا یک فرهنگ است, یک خلاقیت بزرگ, یک پروژه گروهی, در نوبه خود مهم ترین و برجسته ترین. فدورا یک جامعه از مردم است.

اگر در حال خواندن این راهنما هستید, ممکن است تصمیم گرفته باشید که از فضای سیستم عامل های  \index{ویندوز} ویندوز  و \index{مک}مک  دور شوید و یا شاید اخیرا فدورا را بر روی رایانه تان نصب کرده اید, اما مطمئن نیستید که از کجاباید شروع کنید.

استفاده از یک سیستم عامل جدید میتواند ترسناک باشد, مخصوصا وقتی که با کلمه های ناآشنا روبرو میشوید. بسیاری از مردم با اصطلاحات فنی یک سیستم عامل آشنا نیستند و معتقدند که این مفاهیم برایشان خیلی پیشرفته است. در واقع این موضوع درست نیست. فدورا به راحتی نصب میشود و استفاده از آن ساده است. و از همه مهم تر ایتکه: کاملا آزاد و رایگان است.

این راهنما برای کسانی است که به تازگی استفاده از گنو/لینوکس \index{گنو/لینوکس} را شروع کرده اند و این امکان را به آنها میدهد که تمام ابزارهای موردنیاز را بشناسند و از آن ها به درستی استفاده کنند.

شما باخواندن این کتاب میآموزید که چگونه کارهای زیر را انجام دهید:
\begin{itemize}
	\item[\checkmark]
	نصب و راه اندازی فدورا بر روی رایانه تان
	\item[\checkmark]
آشنایی با محیط xfce
	\item[\checkmark]
آموزش نصب و پیکربندی انواع سخت افزار
	\item[\checkmark]
	آشنایی با امکانات شبکه و پیکربندی آن
	\item[\checkmark]
	استفاده از نرم افزارهای سازگار با فدورا
\end{itemize}
