\chapter{معرفی فدورا}\label{ch-1}
قبل از اینکه شروع به نصب کنیم, بهتر است در مورد فلسفه و مفهوم کلی سیستم عامل فدورا صحبت کنیم.\cite{ubuntu}
\section{تعاریف کلی}\label{se-11}
بهتر است قبل از پرداختن به فدورا, در مورد برخی تعاریف, مثل سیستم عامل و هسته, توضیح داده شود.
\subsection{سیستم عامل چیست؟}\label{se-111}
سیستم عامل برنامه ای است که با سخت افزار ارتباط مستقیم دارد و امکان اجرای برنامه های کاربردی را روی بستر سخت افزاری ممکن میسازد.
\subsection{هسته چیست؟}\label{se-112}
هسته نقش قسمت مرکزی و سطح پایین یک سیستم عامل را ایفا میکند و وظایفی مانند ارتباط با سخت افزار و بارگذاری درایورها را به دوش میکشد.
\subsection{لینوکس چیست؟}\label{se-113}
برخلاف تصور خیلی از افراد, لینوکس تنها یک هسته با همان وظایف گفته شده است. بسیار کم پیش می آید که در کاربرد روزانه, به طور مستقیم با خود هسته لینوکس ارتباط برقرار کنیم. با این حال, نقش اصلی را در سیستم عامل بر عهده دارد.
\subsection{چرا گنو/لینوکس آری, لینوکس نه؟}\label{se-114}
همانطور که گفته شد, لینوکس تنها یک هسته است و یک هسته به خودی خود هیچ کاری نمی تواند برای ما انجام دهد. ما برای برطرف کردن نیازهای روزانه به نرم افزارهای متعددی نیازمندیم. نرم افزارهایی که اکثرا از پروژه گنو گرفته شده یا بر اساس فلسفه نرم افزارهای آزاد و مجوز GNU GPL
\footnote{پیشنهاد میکنم لینک کوتاه شده روبرو را در ارتباط با تاریخچه مطالعه بکنید
\href{http://www.linuxfedora.ir/viewtopic.php?f=47&t=132}{http://opizo.com/JSsagT}
}
ساخته شده اند. برای همین, بهتر است این سیستم عامل (و نه خود هسته) را گنو/لینوکس بنامیم.
\index{گنو/لینوکس}
\subsection{نرم افزارهای اختصاصی در مقابل نرم افزار آزاد و منبع باز}\label{se-115}
نرم افزارهای اختصاصی توسط یک شرکت طراحی میشوند, توسعه می یابند و فروخته میشوند. این نرم افزارها برای به دست آوردن سود فروخته میشوند و اکثرا فقط بر روی یک نوع از کامپیوترها کاربرد دارند. برای مثال, سیستم عامل های اختصاصی مانند windows و یا mac را در نظر بگیرید. کد منبع این سیستم ها در دسترس نیست و اگر شما سعی به تغییر یا توزیع آن را داشته باشید, متهم خواهید شد.
فدورا, به عبارت دیگر, یک نرم افزار اختصاصی نیست; به این دلیل که به صورت فعال توسط جامعه FOSS نکهداری میشود.
\subsection{Foss چیست؟}\label{se-116}
FOSS
\index{FOSS}
\footnote{FOSS مخفف عبارت Free/Libre Open Source Software و به معنی نرم افزار آزاد و متن باز است.}
نرم افزار FOSS به دلایل زیر با نرم افزارهای اختصاصی تفاوت دارد:
\begin{itemize}
	\item استفاده آزاد و رایگان
	\item اشتراک گذاری آزاد و رایگان
	\item توسعه آزاد و رایگان
\end{itemize}
این یعنی شما بدون پرداخت هیچ مبلغی میتوانید فدورا را دانلود و استفاده کنید. شما میتوانید به صورت کاملا قانونی از cd/dvd های فدورا به هر تعداد که میخواهید کپی کرده و بین دوستان و آشنایان تان توزیع کنید. حتا کد منبع سیستم عامل فدورا آزادانه در دسترس شماست و میتوانید آن را با توجه به نیازهای خود تغییر دهید.
فدورا از مجوز عمومی همگانی GNU 
\emph{یا به طور ساده GPL}
  استفاده میکند که به طور گسترده در جامعه FOSS استفاده میشود. به همین دلیل, فدورا دارای آزادی هایی است که ذکر شد.
\footnote{پیشنهاد میکنم لینک کوتاه شده روبرو را در ارتباط با نرم افزارهای متن باز را مطالعه کنید
\href{http://www.linuxfedora.ir/viewtopic.php?f=47&t=133}{http://opizo.com/coy6qY}
}
\section{چطور ممکن است فدورا رایگان باشد؟}\label{se-117}
شما ممکن است تعجب کنید که در حال حاضر واقعا چطور ممکن است فدورا رایگان باشد. آیا نکته و یا برخی هزینه های مخفی وجود دارد؟ به دو دلیل فدورا رایگان است.
\subsection{مدیریت و بودجه بندی به پشتوانه ردهت}\label{se-1171}
اگرچه فدورا توسط جامعه FOSS نگهداری میشود, ولی مدیریت و تامین بودجه از طریق شرکت خصوصی ردهت انجام میشود.
ردهت پشتیبانی های تجاری را برای شرکتها تامین میکند و از این راه درآمد دارد. درآمد حاصل از این پشتیبانی, برای توسعه مستمر محصولات ردهت مصرف میشود. این توسعه مستمر, شامل موارد زیر است:
\begin{itemize}
	\item انتشار نسخه های جدید فدورا هر شش ماه یکبار
	\item به روزآوری های امنیتی
	\item سرورهای میزبانی وب برای جامعه آنلاین فدورا
	\item دفاتر فدورا
\end{itemize}
\subsection{فدورا از طریق جامعه FOSS نگهداری میشود}\label{se-1172}
از آنجایی که فدورا نرم افزاری کدباز است, کاربران برای دسترسی و تغییر کد منبع آزاد هستند و این به بهترین شدن سیستم عامل برای همه کمک میکند.

فدورا هم یک جامعه جهانی است و هم یک پروژه نرم افزاری مشترک. مردم در سرتاسر جهان میتوانند زمان و توانایی های خود را با هم به اشتراک بگذارند و در فعالیتهایی مانند زیر کمک کنند:
\begin{itemize}
	\item تست اشکالات نرم افزاری
	\item ارسال مستندات کاربری
	\item ارائه بازخورد
	\item طراحی اثر هنری
\end{itemize}
\section{چرا باید فدورا را استفاده کرد؟}\label{se-118}
\begin{itemize}
	\item کار با فدورا ساده است.
	\item نصب نرم افزار, به روزرسانی سیستم عامل و پیدا کردن ابزارهای جدید با چند کلیک انجام پذیر است.
	\item محیط اصلی فدورا که گنوم نام دارد, بسیار زیباست.
	\item آزاد است و برای همیشه رایگان باقی میماند.
	\item فدورا از هسته لینوکس استفاده میکند که طراحی بسیار منطقی و امنی دارد.
	\item فدورا به طور معمول ویروس نمیگیرد.
	\item  فدورا با اکثر رایانه ها و لپتاپ ها کار میکند و در بیشتر مواقع حتی نیاز به نصب یک درایور هم ندارید.
	\item با برنامه و فایل های فعلی تان سازگار است. اکثر محتوای چند رسانه ای در فدورا قایل پخش است و بسیاری از برنامه ها, مثال فایرفاکس, کروم و تلگرام, نسخه ای مناسب فدورا دارند.
	\item از بسیاری از زبان ها, از جمله زبان فارسی, به خوبی پشتیبانی میکند
	\item پایدار و سرعت آن بالاست. فدورا کند نمیشود و لازم نیست هرچند وقت دوباره نصب اش کنیدو به چندین گیگابایت رم برای اجرا نیاز ندارید.
	\item همیشه کسی برای کمک هست. 
	\footnote{گروه پرسش و پاسخ فدورا در تلگرام
\href{https://t.me/joinchat/DSbv30PVmG2lrcJjsAhwCQ}{http://opizo.com/pPkVlQ}	
}
\footnote{کانال اختصاصی این کتاب در پیام رسان گپ
\href{https://Gap.im/itmans}{https://Gap.im/itmans}
}
      \footnote{در صورت هرگونه سوال یا مشکل میتوانید به گروه پرسش و پاسخ همین کتاب در پیام رسان گپ
	  مراجعه کنید
	  \href{https://gap.im/gnu\_linux}{https://gap.im/gnu\_linux}    
}
\end{itemize}
\section{نکاتی درباره فدورا}\label{se-119}
فدورا مشهور به تمرکز بر روی نوآوری و ادغام فناوری های جدید در لینوکس است.فدورا یک چرخه عمر نسبتا کوتاه دارد:هر نسخه معمولا ۱۳ ماه پشتیبانی میشود, در حالی که نسخه \textbf{X} تنها تا یک ماه پس از انتشار نسخه \textbf{X+2} و تقریبا ۶ ماه از بیشتر نسخه ها پشتیبانی میشود. کاربران فدورا میتوانند بدون نصب مجدد از نسخه به نسخه ارتقاء پیدا کنند.
\cite{wiki-fedora}